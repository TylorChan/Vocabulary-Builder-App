\documentclass{cv}
\usepackage[utf8]{inputenc}
\usepackage[top=0.5in, left=0.5in, right=0.5in, bottom=0.5in]{geometry}
\usepackage{hyperref}
\usepackage{svg}
\usepackage{array}


\newcommand*{\labelfont}{\fontfamily{bch}\selectfont}
\newcommand*{\subfont}{\fontfamily{lmss}\selectfont}

\hypersetup{
    colorlinks=true,
    linkcolor=blue,
    filecolor=blue,      
    urlcolor=blue,
}


\begin{document}

\begin{center}
    {\LARGE \textbf{Nikash Walia} \par}
    \vspace{5pt}
    408-888-9568 $|$ \href{mailto:nikash.walia@gmail.com}{nikash.walia@gmail.com} $|$ \href{nikwalia.github.io}{nikwalia.github.io} $|$ \href{linkedin.com/in/nikash-walia}{linkedin.com/in/nikash-walia}
\end{center}\vspace*{-5pt}

{\large {\textbf{Education}}}\vspace*{-6pt}\\
\rule{\textwidth}{0.4pt}
{\labelfont \textbf{Bachelor and Master of Science in Computer Science} \hspace{6cm} Aug 2019-May 2023}\\
{\small University of Illinois at Urbana-Champaign (GPA-4.0/4.0) \hspace{5.5cm} Advisor: Svetlana Lazebnik\\
Completed: \hspace{0.5cm} Linear Algebra, Numerical Methods, Probability \& Statistics, Algorithms \& Models of Computation, Deep \\\hspace*{2.55cm}Learning, AI for Robot Control, Machine Learning, Computational Photography, Computer Vision\\
Current: \hspace{0.95cm} Efficient and Predictive Computer Vision, Research Independent Study}

\vspace*{5pt}
{\large {\textbf{Skills}}}\vspace*{-6pt}\\
\rule{\textwidth}{0.4pt}
{
{\labelfont \textbf{Languages}}: Python, C, C++, CUDA, Bash/Shell, Java\\
{\labelfont \textbf{ML}}: PyTorch, TensorFlow, Keras, SciKit-Learn, OpenCV, DGL, StellarGraph, Habitat, cuDF, Numba\\
{\labelfont \textbf{Other}}: Linux, Git, Docker, Make, AWS, GCP, Flask, Latex, Neo4J, MySQL
}

\vspace*{5pt}
% \noindent\rule{\textwidth}{1pt}\vspace*{5pt}
{\large {\textbf{Experience}}}\vspace*{-6pt}\\
\rule{\textwidth}{0.4pt}

{\labelfont \textbf{Data Science Intern} \hspace*{10.6cm} \textbf{Intuit Inc.} $|$ May-Aug 2022}\\
\vspace{-25pt}\\
{
\small
\begin{itemize}
    \item Incoming intern. Contributing to document parsing research using OCR, NLP, and transformers.
\end{itemize}
}

{\labelfont \textbf{Product Development Intern} \hspace*{8.5cm} \textbf{VMware, Inc.} $|$ May-Aug 2021}\\
\vspace*{-25pt}\\
{
\small
\begin{itemize}
    \setlength\itemsep{-2.5pt}
    \item Developed graph algorithms based on state-of-the-art models for security rule assignment for virtual machines.
    \item Engineered recommendation pipelines for downstream security group collapse to reduce network complication.
    \item Produced models with 95\%+ validation accuracy. Patent application Docket No. H872.01.
\end{itemize}
}

{\labelfont \textbf{Innovation Lab Intern} \hspace*{6.5cm} \textbf{Caterpillar Inc.- Cat Digital} $|$ Aug 2020-Jan 2021}\\
\vspace*{-25pt}\\
{
\small
\begin{itemize}
    \setlength\itemsep{-2.5pt}
    \item Identified poor dependencies and redesigned code base for Weibull failure analysis.
    \item Developed data aggregation and analysis pipeline using Cyclone and MySQL for clients' business goals.
\end{itemize}
}

{\labelfont \textbf{Data Science Intern} \hspace*{10.15cm} \textbf{Walmart Labs} $|$ Jun-Aug 2020}\\
\vspace*{-25pt}\\
{
\small
\begin{itemize}
\setlength\itemsep{-2.5pt}
    \item Built a pipeline from scratch using TensorFlow to obtain network attention regions for images.
    \item Generated Faster-RCNN embeddings on previously-unseen objects to develop and productionize image-based search tools. Produced manually-tested top-5 accuracy of 80\%. Patent pending.
\end{itemize}
}

{\large {\textbf{Projects}}}\vspace*{-6pt}\\
\rule{\textwidth}{0.4pt}
{\labelfont \textbf{Caterpillar Data Science Challenge, HackIllinois 2020 (Winner)} \hspace*{5.9cm} Mar 2020}
{
\small
\begin{itemize}
\setlength\itemsep{-2.5pt}
    \item Used time series sensor data from HDF files to find anomalies and robustly predict future faults via random forests.
    \item Built a pipeline for efficiently processing data and performing unsupervised anomaly detection using DeepAnT.
\end{itemize}
}
% {\labelfont \textbf{Semantic Segmentation for Object Identification} \hspace*{8.75cm} Dec 2019}
% {
% \small
% \begin{itemize}
% \setlength\itemsep{-2.5pt}
%     \item Designed a pipeline from scratch in C++ to perform instance segmentation on simple, unknown images.
%     \item Used traditional CV techniques using OpenCV to partition and categorize objects without any priors.
% \end{itemize}
% }

% \noindent\rule{\textwidth}{1pt}\vspace*{5pt}
{\large {\textbf{Research}}}\vspace*{-6pt}\\
\rule{\textwidth}{0.4pt}

{\labelfont \textbf{Undergraduate Researcher, CV/RL} \hspace*{6.5cm} \textbf{Svetlana Lazebnik} $|$ July 2021-Present}\\
\vspace*{-25pt}\\
{
\small
\begin{itemize}
\setlength\itemsep{-2.5pt}
    \item Exploring alternative strategies to exploration for reinforcement learning agents across multiple tasks.
    \item Combining auxiliary tasks in the embodied-AI space for improved environment understanding and social learning.
\end{itemize}
}

{\labelfont \textbf{Undergraduate Researcher, AI/HPC} \hspace*{5cm} \textbf{Wen-mei Hwu, IBM C3SR} $|$ Aug 2020-Present}\\
\vspace*{-25pt}\\
{
\small
\begin{itemize}
\setlength\itemsep{-2.5pt}
    \item Building custom extensions for state-of-the-art deep learning models with PyTorch and BERT.
    \item Integrating CUDA kernels to produce speedups for sparse-matrix multiplications. 
\end{itemize}
}

% {\labelfont \textbf{Self-Supervision for Reinforcement Learning} \hspace*{4cm} \textbf{Independent Research} $|$ Jan 2021-Present}\\
% \vspace*{-25pt}\\
% {
% \small
% \begin{itemize}
% \setlength\itemsep{-2.5pt}
%     \item Developing strategies to improve Sequence-to-Sequence RL bots in the Matterport3D environment using self-teaching and curiosity exploration. Based off the R2R and CVDN challenges.
% \end{itemize}
% }

{\labelfont \textbf{Data Science Research Intern} \hspace*{6cm} \textbf{NASA Ames Research Center} $|$ Jun-Aug 2019}\\
\vspace*{-25pt}\\
{
\small
\begin{itemize}
\setlength\itemsep{-2.5pt}
    \item Collaborated with subject-matter experts in astrophysics for exoplanet detection and validation using TensorFlow.
    \item Developed augmented datasets and modified state-of-the-art AstroNet for Kepler Space Telescope data.
    \item Benchmarked to improve accuracy by 3-4\% at internship time. Published in American Astronomical Journals. \url{https://arxiv.org/abs/2111.10009}
\end{itemize}
}

{\large {\textbf{Honors and Certifications}}}\vspace*{-6pt}\\
\rule{\textwidth}{0.4pt}
{\labelfont \textbf{NVIDIA Deep Learning Institute}}: Fundamentals of Accelerated Computing with CUDA Python.\\
{\labelfont \textbf{ISUR Scholar}}: grant for funding independent undergraduate research.\\
{\labelfont \textbf{C3SR-URAI Scholar}}: merit-based selection into research program for systems research.\\
{\labelfont \textbf{UIUC Dean's List}}: awarded for academic excellence.\\
% {\labelfont \textbf{President's Volunteer Service Award}}: for promoting robotics education in regional schools.


\end{document}
